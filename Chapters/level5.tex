\chapter{Level 5: Critical evaluation and reflection}

In this project, we built a network, generated and analysed traffic, conducted network attacks, and implemented defence mechanisms. Our hands-on approach allowed us to gain practical experience and valuable insights into network security.

We began by constructing a network topology consisting of an attacker machine and two victim machines. We verified connectivity and established a baseline understanding of normal network behaviour using tools such as iperf and Wireshark. This foundation was crucial for comparing the impact of network attacks in subsequent stages.

Next, we launched TCP SYN Flood and ICMP Flood attacks from the attacker machine, targeting the victim machines. Through packet and flow analysis in Wireshark, we observed the detrimental effects of these attacks on the network. To mitigate the impact, we implemented iptables firewalls on the victim machines, configuring rules to limit malicious traffic and maintain stable network performance.

Each team member played a vital role in the success of this project. Ahmed Bedair took on the role of the attacker, utilising tools like hping3 to simulate realistic attack scenarios. His expertise was instrumental in demonstrating the potential impact of network attacks.

Yukesh Shrestha and Varishdan Kumariah, acting as victims, were responsible for implementing the network defence mechanisms. Their technical skills in configuring iptables and fine-tuning firewall rules were essential in mitigating the effects of the attacks and ensuring the network's stability and security.

We collaborated effectively Throughout the project, leveraging our strengths and knowledge to overcome challenges. We gained a deeper understanding of network vulnerabilities, attack techniques, and the importance of proactive defence measures.